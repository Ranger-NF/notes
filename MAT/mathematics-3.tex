\documentclass{article}

\setlength{\parindent}{0pt} 

\usepackage{enumitem}
\usepackage{amsmath}
\usepackage{amssymb}


\newlist{qanda}{enumerate}{1}
\setlist[qanda,1]{label=\textbf{Q\arabic*:}, 
                  left=0pt, 
                  itemsep=1em, 
                  align=left, 
                  wide=0pt}

\newcommand{\answer}[1]{\\\textbf{A:} #1}

\title{Mathematics for Information Science - 3}
\author{Mohammed Fahad}

\begin{document}

\maketitle

\section{Experiment \& Subspace}
An experiment is any activity or process whose outcomes are subject to uncertainty.

Sample space denoted by $S$ is the set of all outcomes of that experiment.

\section{Random Variable}
A random variable is a function whose domain is the sample space, $S$ and whose range is the set if real number $R$.

Random variable, 
\[
rv: S \rightarrow R
\]

There are 2 types of random variables:
\begin{enumerate}
    \item Discrete random variable: Whose values constitute a countable set
    \item Continous random variable
\end{enumerate}

\section{Propability distribution/Propability Mass Function (PMF)}
PMF of a discrete random variable is defined for every number, $x$ by:
\[
p(x) = p(X = x)
\]
Satisfying the following conditions:
\begin{enumerate}
    \item \( p(x) \geq 0 \)
    \item \( \sum p(x) = 1 \)
\end{enumerate}

NB: Propability can never be $>$ 1


\textbf{Q.} Check whether the following are PMF
\begin{align*}
    P(x) = \frac{x^2}{25}; x = 0, 1, 2, 3, 4
\end{align*}

\textbf{A.} 
\begin{align*}
    P(0) = \frac{0^2}{25} = 0 \\
    P(1) = 1/25 \\
    P(2) = 4/25 \\
    P(3) = 9/25 \\
    P(4) = 16/25 \\
    P(0) + P(1) + P(2) + P(3) + P(4) = \frac{30}{25} \neq 1 
\end{align*}

Therefore, $P(x)$ can't be a PMF

\end{document}