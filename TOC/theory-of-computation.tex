\documentclass{article}

\usepackage{enumitem}


\newlist{qanda}{enumerate}{1}
\setlist[qanda,1]{label=\textbf{Q\arabic*:}, 
                  left=0pt, 
                  itemsep=1em, 
                  align=left, 
                  wide=0pt}

\newcommand{\answer}[1]{\\\textbf{A:} #1}

\title{Theory of Computation}
\author{Mohammed Fahad}

\begin{document}

\maketitle

\section{Why TOC?}

\begin{itemize}
    \item It helps us to understand the limits of what computer can do and how to model computation using mathematics
\end{itemize}

\begin{qanda}
    \item What is the motivation for studying theory behind computation? OR Needs of TOC?
    \answer{
        \begin{itemize}
            \item Understanding the capability of a computer
            \item To find steps to solve a problem
            \item Increase efficiency while doing a task
        \end{itemize}
    }

    \item List the problems that cannot be solved by a computer.
    \answer{
        \begin{enumerate}
            \item Ethical problems. Eg: Self-driving car deciding to save the driver/passenger or the pedestrian
        \end{enumerate}
    }

\end{qanda}

\section{Automaton (pl.: Automata)}

A simplified mathematical model of a machine (digital computer) - It accepts input, produces output, may have some temporary storage and can mae descisions in transforming the input into the output


\end{document}
