\documentclass{article}
\usepackage{amsmath}

\title{Digital Electronics And Logic Design}
\author{Mohammed Fahad}

\begin{document}

\maketitle

\section*{Syllabus}

Introduction to digital Systems :- Digital abstraction
Number Systems – Binary, Hexadecimal, grouping bits, Base conversion;
Binary Arithmetic – Addition and subtraction, Unsigned and Signed
numbers; Fixed-Point Number Systems; Floating-Point Number Systems
Basic gates- Operation of a Logic circuit; Buffer; Gates - Inverter, AND gate,
OR gate, NOR gate, NAND gate, XOR gate, XNOR gate; Digital circuit,
operation - logic levels, output dc specifications, input dc specifications,
noise margins, power supplies; Driving loads - driving other gates, resistive
loads and LEDs.

\section{Conversion from Decimal to Other Bases}

\subsection{Changing Base of the Whole Number Part}

\[
(18)_{10} = (?)_2
\]

We use repeated division by the target base (2 in this case):

\begin{equation}
\begin{array}{r|l}
18 & \div 2 \\
\hline
9 & \text{remainder } 0 \\
\div 2 \\
\hline
4 & \text{remainder } 1 \\
\div 2 \\
\hline
2 & \text{remainder } 0 \\
\div 2 \\
\hline
1 & \text{remainder } 0 \\
\div 2 \\
\hline
0 & \text{remainder } 1 \\
\end{array}
\end{equation}

Now, read the remainders from **bottom to top** to get the binary equivalent:

\[
(18)_{10} = (10010)_2
\]


\subsection{Changing Base of the Fractional Part}

How do we convert a **fractional decimal** to another base?

\[
(0.25)_{10} = (?)_2
\]

We use repeated multiplication of the fractional part by the base (2 in this case):

\begin{align*}
0.25 \times 2 &= 0.5 \quad \Rightarrow \text{Digit: } 0 \\
0.5 \times 2  &= 1.0 \quad \Rightarrow \text{Digit: } 1 \ (\text{stop: fraction is now 0})
\end{align*}

Now write the digits from top to bottom after the binary point:

\[
(0.25)_{10} = (0.01)_2
\]

\subsection*{Hexadecimal (Base 16)}
\begin{center}
\begin{tabular}{|c|c|}
\hline
\textbf{Decimal} & \textbf{Hexadecimal} \\
\hline
0  & 0 \\
1  & 1 \\
2  & 2 \\
3  & 3 \\
4  & 4 \\
5  & 5 \\
6  & 6 \\
7  & 7 \\
8  & 8 \\
9  & 9 \\
10 & A \\
11 & B \\
12 & C \\
13 & D \\
14 & E \\
15 & F \\
\hline
\end{tabular}
\end{center}

Just like how 10 comes after 9, 10 comes after FF

\end{document}
