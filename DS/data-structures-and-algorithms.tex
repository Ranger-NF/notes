\documentclass{article}

\usepackage{listings}
\usepackage{xcolor}

\lstset{
  language=C,
  basicstyle=\ttfamily\small,
  keywordstyle=\color{blue},
  commentstyle=\color{gray},
  stringstyle=\color{green!50!black},
  numbers=left,
  numberstyle=\tiny,
  stepnumber=1,
  numbersep=8pt,
  showstringspaces=false,
  tabsize=4,
  breaklines=true,
  frame=single,
  captionpos=b
}

\title{Data Structures and Algorithms}
\author{Mohammed Fahad}

\begin{document}

\maketitle

\section*{Syllabus}

\textbf{Basic Concepts of Data Structures}

Definitions; Data Abstraction; Performance Analysis - Time \& Space
Complexity, Asymptotic Notations; Polynomial representation using
Arrays, Sparse matrix (Tuple representation); Stacks and Queues - Stacks,
Multi-Stacks, Queues, Circular Queues, Double Ended Queues; Evaluation
of Expressions- Infix to Postfix, Evaluating Postfix Expressions.

\section{Definitions}
\begin{itemize}
    \item \textbf{Data Structures:} ways of organizing and storing data in a computer so that it can be accessed and modified efficiently. Types:
    \begin{enumerate}
        \item Linear: Arrays, Linked Lists, Stacks, Queues
        \item Non-linear: Trees, Graphs
    \end{enumerate}

    \item \textbf{Data Abstraction:} concept of hiding the internal details of how data is stored or maintained and only showing the essential features or operations that can be performed on the data.
\end{itemize}

\section{Performance Analysis}
\begin{enumerate}
    \item \textbf{Time Complexity}
    \item \textbf{Space Complexity}
\end{enumerate}

\section{Stack}
It follow FILO (First In Last Out) scheme
\begin{itemize}
    \item pop - Removes from top
    \item push - Adds to top
    \item peek/top - See topmost element
\end{itemize}


\section{Queue}
It follow FIFO (First In First Out) scheme
\begin{itemize}
    \item pop - Removes from front
    \item push - Adds to rear
    \item peek/top - See frontmost element
\end{itemize}

\section{Addition of sparse polynomial}

All the polynomials are stored inside an array of structures:
\begin{lstlisting}[caption={Sparse Polynomial Addition Outline}]
    // Structure to represent a term
    typedef struct {
        int coeff;
        int expo;
    } Term;

    Term polynomial[] = {{2, 3},{4, 0}} // 2x^3 + 4
\end{lstlisting}

\subsection{Logic when adding 2 polynomials}
let it be poly1 (with i as indexing),  poly2 (with j as indexing) \&  result (with k as indexing)

\begin{itemize}
    \item If poly1[i].exp ==  poly2[j].exp: add coefficients
    \item If poly1[i].exp is greater than poly2[j].exp: Copy over poly1[i]
    \item If poly1[i].exp is less than poly2[j].exp: Copy over poly2[j]
\end{itemize}


\end{document}
