\documentclass{article}

\usepackage{listings}
\usepackage{xcolor}
\usepackage{amsmath}

\lstset{
  language=C,
  basicstyle=\ttfamily\small,
  keywordstyle=\color{blue},
  commentstyle=\color{gray},
  stringstyle=\color{green!50!black},
  numbers=left,
  numberstyle=\tiny,
  stepnumber=1,
  numbersep=8pt,
  showstringspaces=false,
  tabsize=4,
  breaklines=true,
  frame=single,
  captionpos=b
}

\setlength{\parindent}{0pt} 

\title{Data Structures and Algorithms}
\author{Mohammed Fahad}

\begin{document}

\maketitle

\section*{Syllabus}

\textbf{Basic Concepts of Data Structures}

Definitions; Data Abstraction; Performance Analysis - Time \& Space
Complexity, Asymptotic Notations; Polynomial representation using
Arrays, Sparse matrix (Tuple representation); Stacks and Queues - Stacks,
Multi-Stacks, Queues, Circular Queues, Double Ended Queues; Evaluation
of Expressions- Infix to Postfix, Evaluating Postfix Expressions.

\section{Definitions}
\begin{itemize}
    \item \textbf{Data Structures:} ways of organizing and storing data in a computer so that it can be accessed and modified efficiently. Types:
    \begin{enumerate}
        \item Linear: Arrays, Linked Lists, Stacks, Queues
        \item Non-linear: Trees, Graphs
    \end{enumerate}

    \item \textbf{Data Abstraction:} concept of hiding the internal details of how data is stored or maintained and only showing the essential features or operations that can be performed on the data.
\end{itemize}

\section{Performance Analysis}
\begin{enumerate}
    \item \textbf{Time Complexity}
    \item \textbf{Space Complexity}
\end{enumerate}

\section{Stack}
It follow FILO (First In Last Out) scheme
\begin{itemize}
    \item pop - Removes from top
    \item push - Adds to top
    \item peek/top - See topmost element
\end{itemize}

\subsection{Implementation of Stack}

\begin{lstlisting}[caption={Implementation of Stack}]
#include <stdio.h>
#define MAX 2

int stack[MAX];
int top = -1;

void push(int a) {
    if (top + 1 >= MAX) {
        printf("Stack Overflow\n");
    } else {
        stack[++top] = a;
    }
}

int pop() {
    if (top == -1) {
        printf("Stack underflow \n");
        return -1;
    } else {
        return stack[top--];
    }
}

void display() {
    if (top == -1) {
        printf("Stack is empty!\n");
    } else {
        for (int i = top; i > -1; i--) {
            printf("%d ", stack[i]);
        }
        printf("\n");
    }
}

int main() {
    pop();

    push(5);
    push(8);

    display();

    pop();

    display();

    return 0;
}
\end{lstlisting}


\section{Queue}
It follow FIFO (First In First Out) scheme
\begin{itemize}
    \item pop - Removes from front
    \item push - Adds to rear
    \item peek/top - See frontmost element
\end{itemize}

\subsection{Implementation of Queue}

\begin{lstlisting}[caption={Implementation of Queue}]
#include <stdio.h>
#define MAX 3

int queue[MAX];
int rear = -1, front = -1;

void enqueue(int a) {
    if (rear + 1 >= MAX) {
        printf("Queue Overflow\n");
    } else {
        if (front == -1) front = 0;
        queue[++rear] = a;
    }
}

int dequeue() {
    if (front > rear) {
        printf("Queue underflow \n");
        return -1;
    } else {
        return queue[front++];
    }
}

void display() {
    if (rear == -1 || front > rear) {
        printf("Queue is empty!\n");
    } else {
        for (int i = rear; i >= front; i--) {
            printf("%d ", queue[i]);
        }
        printf("\n");
    }
}
\end{lstlisting}

\section{Multi-Stacks}
2 or more stacks in a single array.

\subsection{2 stacks in one array}

Stack 1 grows from left to right.
Stack2 grows frome right to left. \textbf{Condition:}

\begin{itemize}
    \item To push into stack1: \[top1 + 1 < top2\]
    \item To push into stack2: \[top2 - 1 > top1\]
\end{itemize}

\section{Addition of sparse polynomial}

All the polynomials are stored inside an array of structures:
\begin{lstlisting}[caption={Sparse Polynomial Addition Outline}]
    // Structure to represent a term
    typedef struct {
        int coeff;
        int expo;
    } Term;

    Term polynomial[] = {{2, 3},{4, 0}} // 2x^3 + 4
\end{lstlisting}

\subsection{Logic when adding 2 polynomials}
let it be poly1 (with i as indexing),  poly2 (with j as indexing) \&  result (with k as indexing)

\begin{itemize}
    \item If poly1[i].exp ==  poly2[j].exp: add coefficients
    \item If poly1[i].exp is greater than poly2[j].exp: Copy over poly1[i]
    \item If poly1[i].exp is less than poly2[j].exp: Copy over poly2[j]
\end{itemize}

\section{Sparse Matrix}
Sparse matrix is a matrix with most of its elements are zero. Eg:
\[
\begin{matrix}
15 & 0  & 0  & 0  & 91 & 0  \\
0  & 11 & 0  & 0  & 0  & 0  \\
0  & 3  & 0  & 0  & 0  & 28 \\
22 & 0  & -6 & 0  & 0  & 0  \\
0  & 0  & 0  & 0  & 0  & 0  \\
-15& 0  & 0  & 0  & 0  & 0  \\
\end{matrix}
\]

\subsection{Finding transpose of sparse matrix}

We use an array of structs to save values of non-zero values of matrix.
\begin{lstlisting}[caption={Representing sparse matrix}]
    typedef struct {
        int row;
        int col;
        int value;
    } Term;
\end{lstlisting}

The first row of the array would be the \textbf{metadata} of the matrix: Number of rows, Number of columns and number of non-zero elements.

One such array for representation might look like this:
\begin{lstlisting}
Term a[] = {
    {6, 6, 8}, // metadata: 6 rows, 6 cols, 8 non-zero values
    {0, 0, 15},
    {3, 0, 22},
    {5, 0, -15},
    {1, 1, 11},
    {2, 1, 3},
    {3, 2, -6},
    {0, 4, 91},
    {2, 5, 28}
};
\end{lstlisting}

\subsubsection{Program for transpose}
\begin{lstlisting} [caption=Finding the transpose of a sparse matrix]
void transpose(Term a[], Term b[]) {
    int n = a[0].value;

    b[0].row = a[0].col;
    b[0].col = a[0].row;
    b[0].value = n;

    if (n > 0) {
        int indexb = 1; // To keep track of values in b
        for (int i = 0; i < a[0].col; i++) { // For sorting
            for (int j = 1; j <= n; j++) { // For iteration
                if (a[j].col == i) {
                    b[indexb].row = a[j].col;
                    b[indexb].col = a[j].row;
                    b[indexb].value = a[j].value;
                    indexb++;
                }
            }
        }
    }
}
\end{lstlisting}
\end{document}
